% Para el diagrama del controlador y robot, no reutilizable
\usepackage{tikz}
\usetikzlibrary{arrows,positioning,patterns} 
\tikzset{
    %Define standard arrow tip
    >=stealth',
    %Define style for boxes
    punkt/.style={
        rectangle,
        rounded corners,
        draw, very thick,
        text width=6.5em,
        minimum height=2em,
        text centered},
    % Define arrow style
	pil/.style={ 
		->, 
		thick, 
		shorten <=2pt, 
		shorten >=2pt,} 
} 

% Variables para el robot, no reutilizable
\newcommand{\nvar}[2]{%
    \newlength{#1}
	\setlength{#1}{#2}
}

% Define a few constants for drawing
\nvar{\dg}{0.3cm}
\def\dw{0.25}\def\dh{0.5}
\nvar{\ddx}{1.5cm}

% Define commands for links, joints and such
\def\link{\draw [double distance=1.5mm, very thick] (0,0)--}
\def\joint{%
    \filldraw [fill=white] (0,0) circle (5pt);
    \fill[black] circle (2pt);
}
\def\grip{%
    \draw[ultra thick](0cm,\dg)--(0cm,-\dg);
    \fill (0cm, 0.5\dg)+(0cm,1.5pt) -- +(0.6\dg,0cm) -- +(0pt,-1.5pt);
    \fill (0cm, -0.5\dg)+(0cm,1.5pt) -- +(0.6\dg,0cm) -- +(0pt,-1.5pt);
}
\def\robotbase{%
    \draw[rounded corners=8pt] (-\dw,-\dh)-- (-\dw, 0) --
       (0,\dh)--(\dw,0)--(\dw,-\dh);
    \draw (-0.5,-\dh)-- (0.5,-\dh);
    \fill[pattern=north east lines] (-0.5,-1) rectangle (0.5,-\dh);
}

% Draw an angle annotation
% Input:
%   #1 Angle
%   #2 Label
% Example:
%   \angann{30}{$\theta_1$}
\newcommand{\angann}[2]{%
    \begin{scope}[red]
    \draw [dashed, red] (0,0) -- (1.2\ddx,0pt);
    \draw [->, shorten >=3.5pt] (\ddx,0pt) arc (0:#1:\ddx);
    % Unfortunately automatic node placement on an arc is not supported yet.
    % We therefore have to compute an appropriate coordinate ourselves.
    \node at (#1/2-2:\ddx+8pt) {#2};
    \end{scope}
}

% Draw line annotation
% Input:
%   #1 Line offset (optional)
%   #2 Line angle
%   #3 Line length
%   #5 Line label
% Example:
%   \lineann[1]{30}{2}{$L_1$}
\newcommand{\lineann}[4][0.5]{%
    \begin{scope}[rotate=#2, blue,inner sep=2pt]
    \draw[dashed, blue!40] (0,0) -- +(0,#1)
  	  node [coordinate, near end] (a) {};
	\draw[dashed, blue!40] (#3,0) -- +(0,#1)
	  node [coordinate, near end] (b) {};
	\draw[|<->|] (a) -- node[fill=white] {#4} (b);
	\end{scope}
}

% Define the kinematic parameters of the three link manipulator.
\def\thetaone{30}
\def\Lone{2}
\def\thetatwo{30}
\def\Ltwo{2}
\def\thetathree{30}
\def\Lthree{1}

% Especial para estas diapositivas \usepackage[official]{eurosym}

\definecolor{dkgreen}{rgb}{0,0.6,0}
\definecolor{gray}{rgb}{0.5,0.5,0.5}
\definecolor{mauve}{rgb}{0.58,0,0.82}
 
